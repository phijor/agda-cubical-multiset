\documentclass{easychair}

\usepackage{multiset}
\usepackage{todonotes}
\usepackage{thmtools}

\declaretheorem{theorem}

\begin{document}
  \title{Final Coalgebra of the Finite Bag Functor}
  \titlerunning{Final Coalgebra of the Finite Bag Functor}
  \author{%
      Philipp Joram\inst{1} \and
      Niccol{\`o} Veltri\inst{1}%
  }
  %
  \authorrunning{P. Joram et al.}
  \institute{Department of Software Science, Tallinn University of Technology, Estonia}

  \maketitle

  %\section{Introduction}

  The powerset and multiset (or bag) functor, delivering the set of
  subsets (resp. multisubsets) of a given set, are fundamental tool in
  the behavioural analysis of nondeterministic systems. Such as a
  system can be described as a coalgebra $c : S \to F\, S$, with $F$
  being either powerset or bag functor, associating to each state $s :
  S$ the collection $c \, x$ of reachable states. When $F$ is the the
  bag functor, a state $x$ can reach another state $y$ in multiple
  ways, specified by the multiplicity of $y$ in $c\,x$. One can also
  study systems with finite reachable states and employ finite
  variants of the powerset and bag functor.  The finite bag functor
  takes a set $X$ to the set of finite collections of elements of
  $X$.  In contrast to the finite powerset functor, these collections
  distinguish multiple occurrences of identical elements, e.g.\@ $\Bag[1,
    2, 1]$ is a bag with an element 1 with multiplicity 2.
  
  The behavior of a finitely nondeterministic system starting from a
  given initial state $x : S$ is fully captured by the final coalgebra
  of the collection functor $F$, whose elements are non-wellfounded
  trees obtained by iteratively running the coalgebra $c$ on $x$. In
  recent work \cite{Veltri2021}, we considered various constructions
  of the final coalgebra in the setting of Homotopy Type Theory
  (HoTT). In this work, we explore possible definitions of the finite
  bag functor in HoTT and investigate wheather they admit a final coalgebra.

%%     A coalgebra of $\M$ is a map ${\alpha : X → \M
%%     X}$.  Coalgebras and coalgebra maps form a category, and a final
%%   coalgebra $\omega : V_\omega → \M V_\omega$ defines a greatest
%%   fixpoint $V_\omega ≃ \M V_\omega$.
%% 
%%   A bag functor is an instance of an \emph{analytic} functor \cite{Joyal1986}.
%%   A set-valued functor $F$ is analytic if it is naturally isomorphic to
%%   \[
%%     F(X) \cong \sum_{n : \N} \tfrac{X^n}{G(n)}
%%   \]
%%   where $G(n)$ is a subgroup of the symmetric group $\Sym(s)$ acting
%%   on the projections of the $n$-fold cartesian product $X^n$.
%%   In the case of a bag functor, $G(n)$ is the full permutation group $\Sym(n)$.

  In classical set theory, the final coalgebra of the finite bag
  functor $F$ is obtained as the limit of the
  $\omega^{\operatorname{\mathbf{op}}}$-chain
  \cite[{Ch. 3.3.13}]{Adamek2021}
  \begin{equation}\label{eq:chain}
    1 \xleftarrow{!} {F 1}
      \xleftarrow{F(!)} {F^2 1}
      \xleftarrow{F^2(!)} {F^3 1}
      \xleftarrow{F^3(!)}
      \cdots
  \end{equation}
  where $F^n 1$ is the $n$-th iteration of the functor $F$ on the
  singleton set 1, and $F^n(!)$ is the $n$-th iteration of the
  functorial action of $F$ on the unique function ! targeting the set
  1. In a constructive environment such as HoTT, however, it is not
  immediately obvious whether this construction still produces the
  final coalgebra or not.

%  \subsection{Contributions}

  In this work we give multiple plausible definitions of finite bag
  functors in an univalent setting. We show that the set-valued
  candidates are not adequate for constructing the final coalgebra as
  the limit of the chain (\ref{eq:chain}), as they imply
  non-constructive principles (Theorem \ref{thm:InjPresImpliesLLPO}).
  Instead, we follow \cite{Kock2012} and provide a groupoid-valued
  definition $\FMGpd$ which admits a final coalgebra built as the
  (homotopy) limit of (\ref{eq:chain}) (Theorem \ref{thm:FMGpdLim}).  We
  conclude by comparing these definitions and their final coalgebras.

  We formalized our work in \emph{Cubical Agda} \cite{Vezzosi2019},
  which is a particular implementation of Homotopy Type Theory
  with support for the univalence principle (``equivalence of types is equivalent to equality of types'', which is a provable theorem in Cubical Agda)
  and for higher inductive types.
  \todo[inline]{%
    Show $\SetQuot{}$ or $\SetTrunc{}$,%
    explain what sets and groupoids are.%
    Make sure we only refer to these terms in that way%
  }

  \subsection*{Finite Bags in Sets}
  In a first attempt, we define
  \begin{align*}
    \FMSet X
      \DefEq{}
      \sum\Var{sz} : ℕ\Where
        (\Fin \Var{sz} \to X) \SetQuot \mathrel{\sim_\Var{sz}},
  \end{align*}
  where $(\SymmetricAction{\Var{sz}}[][])$ relates
  $v, w : \Fin \Var{sz} → X$ iff there merely exists a permutation $\sigma$
  such that $v \circ \sigma = w$.
  We show that this definition enjoys a few desirable properties.
  In particular $\FMSet{X}$ is indeed a set, regardless of the homotopy level of $X$.
  Additionally, $\FMSet$ is invariant under set-truncation, i.e.
  $\FMSet \SetTrunc{X} ≃ \FMSet X$.
  The type of finite bags can equivalently defined as the free commutative monoid on $X$
  \cite{Choudhury2021}, which can directly be expressed as a higher inductive type.

  We proceed to prove that this poses a problem when attempting to construct a final coalgebra from the chain mentioned above:
  The relation $(\SymmetricAction{\Var{sz}}[][])$ encodes
  permutations of multisets as property instead of data,
  thus making it impossible to recover information about all terms in a chain
  when proving preservation of limits.
  More precisely, we prove the following theorem in the language of HoTT:

  \begin{theorem}\label{thm:InjPresImpliesLLPO}
    The limit-preservation map
    \[
        \operatorname{pres}\colon
            \FMSet (\lim_{n < \omega} \FMSet^n 1)
            \to
            \lim_{n < \omega} (\FMSet^{n+1} 1)
    \]
    is surjective,
    but its injectivity implies the \emph{lesser limited principle of omniscience}, \LLPO.
  \end{theorem}
  \LLPO{} \cite[{Ch.\@ 1}]{Bridges1987} is a weaker version of the law of the excluded middle, and not
  provable from intuitionistic axioms alone.
  It states that for any (infinite) stream of boolean values that yields $\True$
  at most once, one can decide whether all even or all odd positions are $\False$.
  In particular, the classical construction of the final coalgebra for $\FMSet$
  cannot be replicated in our constructive setting without assumption of classical principles.

  \subsection*{Finite Bags in Groupoids}

  To remedy the situation, we introduce a bag functor that treats identifications of bags as data.
  Let $\FinSet$ be the type of all types merely equivalent to some $\Fin k$:
  \[
    \FinSet \DefEq
      \ldots
  \]
  We define for any type $X$ the type
  \[
    \FMGpd X \DefEq
      \sum_{A : \FinSet} A → X
  \]
  Since $\FinSet$ is a groupoid, each $\FMGpd X$ has a homotopy level of at least that of a groupoid.

  While we usually suppress any size-related issues, they play a crucial r\^{o}le in the formalization.
  Notice that $\FinSet$ is a large type compared to the types it ranges over,
  but it admits small axiomatization as a HIT $\Bij$ as introduced in \cite{Finster2021}:
  \[
    \text{inference rules for $\Bij$}
  \]

  We argue that this is the correct perspective on bags in Homotopy Type Theory,
  and substantiate the claim by the following theorem:

  \begin{theorem}\label{thm:FMGpdLim}
    Let $L_{\FMGpd} \DefEq \lim_{n < \omega} \FMGpd^n 1$.
    The limit-preservation map $\operatorname{pres}$ is an equivalence of groupoids.
    In particular, the limit $L_{\FMGpd}$ is a fixpoint of $\FMGpd$ and the final coalgebra.
  \end{theorem}

  The proof of this is a direct consequence of \cite{Ahrens2015}, since $\FMGpd$ is
  now a polynomial functor in groupoids.
  The existence of $L_{\FMGpd}$ requires $\FMGpd$ to be defined in terms of $\Bij$,
  otherwise it is not typeable in a universe of fixed size.

  \todo[inline]{%
    Where does your work using coinductive datatypes go?
  }

  \subsection*{Comparison}

  By identifying the higher structure of $\FMGpd$, we recover $\FMSet$:
  \begin{theorem}
    For any type $X$, there is an equivalence $\SetTrunc{\FMGpd X} ≃ \FinSet X$.
  \end{theorem}
  Here, $\SetTrunc{\Blank}$ is the \emph{set-truncation} operation which collapses
  higher paths to produce a type of the homotopy level of a set.
  One might wonder if the final coalgebra in groupoids can be used to define a final coalgebra also in sets.
  Indeed, $\FMSet$ has a fixpoint $\FMSet \SetTrunc{L_{\FMGpd}} ≃ \SetTrunc{L_{\FMGpd}}$.
  To further show that this is the final coalgebra seems to require the (full) axiom of choice.

%  \section{Future Work}

  \section*{Acknowledgments}
  This work was supported by the Estonian Research Council grant PSG749.

\bibliographystyle{splncs04}
\bibliography{Multiset}
\end{document}
