\documentclass{easychair}

\usepackage{multiset}
\usepackage{todonotes}
\usepackage{thmtools}

\declaretheorem{theorem}

\begin{document}
  \title{Final Coalgebra of the Finite Bag Functor}
  \titlerunning{Final Coalgebra of the Finite Bag Functor}
  \author{%
      Philipp Joram\inst{1} \and
      Niccolò Veltri\inst{1}%
  }
  %
  \authorrunning{P. Joram et al.}
  \institute{Department of Software Science, Tallinn University of Technology, Estonia}

  \maketitle

  \section{Introduction}

  In this work, we explore possible definitions of the finite bag functor in
  the setting of Homotopy Type Theory, and investigate they admit a final
  coalgebra.
  A finite bag (or multiset) functor $\M$ takes a type $X$ to the type of finite
  collections of terms of $X$.
  In contrast to the finite powerset functor, these collections distinguish
  multiple occurrences of identical terms, e.g.\@ $\Bag[1, 2, 1] : \M X$.
  A coalgebra of $\M$ is a map ${\alpha : X → \M X}$.
  Coalgebras and coalgebra maps form a category, and a final coalgebra
  $\omega : V_\omega → \M V_\omega$ defines a greatest fixpoint
  $V_\omega ≃ \M V_\omega$.

  A bag functor is an instance of an \emph{analytic} functor \cite{Joyal1986}.
  A set-valued functor $F$ is analytic if it is naturally isomorphic to
  \[
    F(X) \cong \sum_{n : \N} \tfrac{X^n}{G(n)}
  \]
  where $G(n)$ is a subgroup of the symmetric group $\Sym(s)$ acting
  on the projections of the $n$-fold cartesian product $X^n$.
  In the case of a bag functor, $G(n)$ is the full permutation group $\Sym(n)$.
  In a classical setting, the final coalgebra of such a functor $F$ is obtained
  from the $\omega^{\operatorname{\mathbf{op}}}$-chain \cite[{3.3.13}]{Adamek2021}
  \[
    1 \xleftarrow{!} {F 1}
      \xleftarrow{F(!)} {F^2 1}
      \xleftarrow{F^2(!)} {F^3 1}
      \xleftarrow{F^3(!)}
      \cdots
  \]
  In a more constructive environment such as Homotopy Type Theory however,
  it is not immediately obvious whether this still holds or not.

  \section{Contributions}

  In this work we give multiple plausible definitions of bag functors in an univalent setting.
  We show that the set-valued candidates are not adequate to define such limits,
  as they imply non-constructive principles (Theorem \ref{thm:InjPresImpliesLLPO}).
  Instead, we follow \cite{Kock2012} and provide a groupoid-valued definition $\FMGpd$ which
  admits a (homotopy) limit of the $\omega^{\operatorname{\mathbf{op}}}$-chain (Theorem \ref{thm:FMGpdLim}).
  We conclude by comparing these definitions and their final coalgebras.

  We formalized our work in \emph{Cubical Agda} \cite{Vezzosi2019},
  which is a particular implementation of Homotopy Type Theory
  with support the univalence principle (as a consequence of the theory)
  and for higher inductive types.
  \todo[inline]{%
    Show $\SetQuot{}$ or $\SetTrunc{}$,%
    explain what sets and groupoids are.%
    Make sure we only refer to these terms in that way%
  }

  \subsection{Finite Bags in Sets}
  In a first attempt, we define
  \begin{align*}
    \FMSet X
      \DefEq{}
      \sum\Var{sz} : ℕ\Where
        (\Fin \Var{sz} \to X) \SetQuot \mathrel{\sim_\Var{sz}},
  \end{align*}
  where $(\SymmetricAction{\Var{sz}}[][])$ relates
  $v, w : \Fin \Var{sz} → X$ iff there merely exists a permutation $\sigma$
  such that $v \circ \sigma = w$.
  We show that this definition enjoys a few desirable properties.
  In particular $\FMSet{X}$ is indeed a set, regardless of the homotopy level of $X$.
  Additionally, $\FMSet$ is invariant under set-truncation, i.e.
  $\FMSet \SetTrunc{X} ≃ \FMSet X$.
  The type of finite bags can equivalently defined as the free commutative monoid on $X$
  \cite{Choudhury2021}, which can directly be expressed as a higher inductive type.

  We proceed to prove that this poses a problem when attempting to construct a final coalgebra from the chain mentioned above:
  The relation $(\SymmetricAction{\Var{sz}}[][])$ encodes
  permutations of multisets as property instead of data,
  thus making it impossible to recover information about all terms in a chain
  when proving preservation of limits.
  More precisely, we prove the following theorem in the language of HoTT:

  \begin{theorem}\label{thm:InjPresImpliesLLPO}
    The limit-preservation map
    \[
        \operatorname{pres}\colon
            \FMSet (\lim_{n < \omega} \FMSet^n 1)
            \to
            \lim_{n < \omega} (\FMSet^{n+1} 1)
    \]
    is surjective,
    but its injectivity implies the \emph{lesser limited principle of omniscience}, \LLPO.
  \end{theorem}
  \LLPO{} \cite[{Ch.\@ 1}]{Bridges1987} is a weaker version of the law of the excluded middle, and not
  provable from intuitionistic axioms alone.
  It states that for any (infinite) stream of boolean values that yields $\True$
  at most once, one can decide whether all even or all odd positions are $\False$.
  In particular, the classical construction of the final coalgebra for $\FMSet$
  cannot be replicated in our constructive setting without assumption of classical principles.

  \subsection{Finite Bags in Groupoids}

  To remedy the situation, we introduce a bag functor that treats identifications of bags as data.
  Let $\FinSet$ be the type of all types merely equivalent to some $\Fin k$:
  \[
    \FinSet \DefEq
      \ldots
  \]
  We define for any type $X$ the type
  \[
    \FMGpd X \DefEq
      \sum_{A : \FinSet} A → X
  \]
  Since $\FinSet$ is a groupoid, each $\FMGpd X$ has a homotopy level of at least that of a groupoid.

  While we usually suppress any size-related issues, they play a crucial r\^{o}le in the formalization.
  Notice that $\FinSet$ is a large type compared to the types it ranges over,
  but it admits small axiomatization as a HIT $\Bij$ as introduced in \cite{Finster2021}:
  \[
    \text{inference rules for $\Bij$}
  \]

  We argue that this is the correct perspective on bags in Homotopy Type Theory,
  and substantiate the claim by the following theorem:

  \begin{theorem}\label{thm:FMGpdLim}
    Let $L_{\FMGpd} \DefEq \lim_{n < \omega} \FMGpd^n 1$.
    The limit-preservation map $\operatorname{pres}$ is an equivalence of groupoids.
    In particular, the limit $L_{\FMGpd}$ is a fixpoint of $\FMGpd$ and the final coalgebra.
  \end{theorem}

  The proof of this is a direct consequence of \cite{Ahrens2015}, since $\FMGpd$ is
  now a polynomial functor in groupoids.
  The existence of $L_{\FMGpd}$ requires $\FMGpd$ to be defined in terms of $\Bij$,
  otherwise it is not typeable in a universe of fixed size.

  \todo[inline]{%
    Where does your work using coinductive datatypes go?
  }

  \subsection{Comparison}

  By identifying the higher structure of $\FMGpd$, we recover $\FMSet$:
  \begin{theorem}
    For any type $X$, there is an equivalence $\SetTrunc{\FMGpd X} ≃ \FinSet X$.
  \end{theorem}
  Here, $\SetTrunc{\Blank}$ is the \emph{set-truncation} operation which collapses
  higher paths to produce a type of the homotopy level of a set.
  One might wonder if the final coalgebra in groupoids can be used to define a final coalgebra also in sets.
  Indeed, $\FMSet$ has a fixpoint $\FMSet \SetTrunc{L_{\FMGpd}} ≃ \SetTrunc{L_{\FMGpd}}$.
  To further show that this is the final coalgebra seems to require the (full) axiom of choice.

  \section{Future Work}

  \section*{Acknowledgments}
  This work was supported by the Estonian Research Council grant PSG749.

\bibliographystyle{splncs04}
\bibliography{Multiset}
\end{document}
