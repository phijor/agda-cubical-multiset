\documentclass[a4paper]{llncs}

\usepackage{multiset}
\usepackage{todonotes}

\begin{document}

  \maketitle

  \section{Introduction}

  In this work, we explore possible definitions of the finite bag functor in
  the setting of Homotopy Type Theory, and investigate they admit a final
  coalgebra.
  A finite bag (or multiset) functor $\M$ takes a type $X$ to the type of finite
  collections of terms of $X$.
  In contrast to the finite powerset functor, these collections distinguish
  multiple occurrences of identical terms, e.g.\@ $\Bag[1, 2, 1] : \M X$.
  A coalgebra of $\M$ is a map ${\alpha : X → \M X}$.
  Coalgebras and coalgebra maps form a category, and a final coalgebra
  $\omega : V_\omega → \M V_\omega$ defines a greatest fixpoint
  $V_\omega ≃ \M V_\omega$.

  A bag functor is an instance of an \emph{analytic} functor \cite{Joyal1986}.
  A set-valued functor $F$ is analytic if it is naturally isomorphic to
  \[
    F(X) \cong \sum_{n : \N} \tfrac{X^n}{G(n)}
  \]
  where $G(n)$ is a subgroup of the symmetric group $\Sym(s)$ acting
  on the projections of the $n$-fold cartesian product $X^n$.
  In the case of a bag functor, $G(n)$ is the full permutation group $\Sym(n)$.
  In a classical setting, the final coalgebra of such a functor $F$ is obtained
  from the $\omega^{\operatorname{\mathbf{op}}}$-chain \cite[{3.3.13}]{Adamek2021}
  \[
    1 \xleftarrow{!} {F 1}
      \xleftarrow{F(!)} {F^2 1}
      \xleftarrow{F^2(!)} {F^3 1}
      \xleftarrow{F^3(!)}
      \cdots
  \]
  In a more constructive environment such as Homotopy Type Theory however,
  it is not immediately obvious whether this still holds.

  \section{Contributions}

  In this work we give multiple plausible definitions of bag functors in an univalent setting.
  We show that the set-valued candidates are not adequate, as they imply non-constructive principles (Theorem \ref{thm:InjPresImpliesLLPO}).
  Instead, we follow \cite{Kock2012} and provide a groupoid-valued definition $\FMGpd$ which
  admits a (homotopy) limit of the $\omega^{\operatorname{\mathbf{op}}}$-chain.
  We conclude by comparing these definitions and their final coalgebras.

  We formalized our work in \emph{cubical Agda}.
  \todo[inline]{
    What exactly should we say about the formalization?
  }

  \subsection{Finite Bags in Sets}
  In a first attempt, we define
  \begin{align*}
    \FMSet X
      \DefEq{}
      \sum\Var{sz} : ℕ\Where
        (\Fin \Var{sz} \to X) \SetQuot \mathrel{\sim_\Var{sz}},
  \end{align*}
  where $(\mathunderscore\sim_{\Var{sz}}\mathunderscore)$ relates
  $v, w : \Fin \Var{sz} → X$ iff there merely exists a permutation $\sigma$
  such that $v \circ \sigma = w$.
  We show that this definition enjoys a few desirable properties.
  In particular $\FMSet{X}$ is indeed a set, regardless of the homotopy level of $X$.
  Additionally, $\FMSet$ is invariant under set-truncation, i.e.
  $\FMSet \SetTrunc{X} ≃ \FMSet X$.
  We proceed to prove that this poses a problem when attempting to construct a final coalgebra from the chain mentioned above:
  The relation $(\SymmetricAction{\Var{sz}}[{}][{}])$ encodes
  permutations of multisets as property instead of data,
  thus making it impossible to recover information about all terms in a chain
  when proving preservation of limits.
  More precisely, we prove the following theorem in the language of HoTT:

  \begin{theorem}\label{thm:InjPresImpliesLLPO}
    Injectivity of the limit-preservation map
    \[
        \operatorname{pres}_{\Lim (\Chain \M)}\colon
            \M (\Lim (\Chain \M))
            \to
            \Lim (\Shift(\Chain\M))
    \]
    implies the \emph{lesser limited principle of omniscience}, \LLPO.
  \end{theorem}
  \todo[inline]{%
    We prove this for the HIT-definition,
    but I state it here for the $\N$-indexed definition.
    Either say both are equivalent or prove it for the other definition.
  }
  \LLPO{} is a weaker version of the law of the excluded middle, and not
  provable from intuitionistic axioms alone.
  It states that for any (infinite) stream of boolean values that yields $\True$
  at most once, one can decide whether all even or all odd positions are $\False$.

  \subsection{Finite Bags in Groupoids}

  To remedy the situation, we introduce a bag functor that treats identifications of bags as data.
  Let $\FinSet$ be the type of all types merely equivalent to some $\Fin k$.
  We define for any type $X$ the type
  \[
    \FMGpd X \DefEq
      \sum_{A : \FinSet} A → X
  \]
  Since $\FinSet$ is a groupoid, each $\FMGpd X$ has a homotopy level of at least that of a groupoid.
  We argue that this is the correct perspective on bags in HoTT, and substantiate the claim by the following theorem:

  \begin{theorem}
    The $\omega^{\mathbf{op}}$-chain
    \[
      1 \xleftarrow{!} {\FMGpd 1}
        \xleftarrow{\FMGpd(!)} {\FMGpd^2 1}
        \xleftarrow{\FMGpd^2(!)} {\FMGpd^3 1}
        \xleftarrow{\hphantom{\FMGpd^3(!)}}
        \cdots
    \]
    has a limit and yields an equivalence of groupoids
    $V_\omega ≃ \FMGpd V_\omega$.
  \end{theorem}
  %
  In particular, this limit induces a fixpoint $\SetTrunc{V_\omega} ≃ \FMSet{\SetTrunc{V_\omega}}$.

  \todo[inline]{%
    Did we show which of these are actually final coalgebras?
    Being a fixpoint alone doesn't mean much.
  }

  While we usually suppress any size-related issues, they play a crucial r\^{o}le in the formalization.
  Since $\FinSet$ is itself a large type compared to the types it ranges over,
  the iteration $\lambda n\Where \FMGpd^n 1$ is not typeable in any universe of fixed size.
  We work around this issue by implementing $\Bij$ as introduced in \cite{Finster2021},
  which is a small type in the base universe, equivalent to $\FinSet$.

  \section{Future Work}

  \section*{Acknowledgments}
  This work was supported by the Estonian Research Council grant PSG749.

\bibliographystyle{splncs04}
\bibliography{Multiset}
\end{document}
