\documentclass[a4paper]{llncs}

\usepackage{multiset}
\usepackage{todonotes}

\begin{document}

  \maketitle

  \section{Contributions}

  In this work, we explore possible definitions of the finite bag functor in
  the setting of Homotopy Type Theory, and investigate they admit a final
  coalgebra.
  A finite bag (or multiset) functor $\M$ takes a type $X$ to the type of finite
  collections of terms of $X$.
  In contrast to the finite powerset functor, these collections may contain
  multiple occurrences of a term $x : X$.
  A coalgebra of $\M$ is a map \linebreak[4] ${\alpha : \M X → X}$.
  Coalgebras and coalgebra maps form a category, and a final coalgebra
  $\omega : \M V_\omega → V_\omega$ defines a greatest fix point
  $\M V_\omega ≃ V_\omega$.

  In a classical setting, a bag functor is an instance of an \emph{analytic}
  functor, i.e.\@ one that preserves countable filtered limits%
    \todo{Is that true? Remember to have an offline copy of the nLab at hand on the next flight},
  and from this it follows that the limit of the chain
  \[
    1 \xleftarrow{!} {\M 1}
      \xleftarrow{\M(!)} {\M^2 1}
      \xleftarrow{\M^2(!)} {\M^3 1}
      \xleftarrow{\hphantom{\M^3(!)}}
      \cdots
  \]
  always exists and forms a final coalgebra.\todo{Is this \cite{Worrell2005}?}

  In a more constructive environment such as Homotopy Type Theory however,
  it is not immediately obvious whether this still holds.
  For one, there exist two definitions of analytic functors that are
  immediately shown to be equivalent in a classical setting:
  the aforementioned one as a limit-preserving functor, and the other as
  functor presented as
  \[
    F(X) ≃ \sum_{n : \N} \tfrac{X^n}{G(n)}
  \]
  where $G(n)$ is some group acting on the projections of the $n$-fold
  cartesian product $X^n$.
  In the case of a (the) bag functor, $G(n)$ is the full permutation group on
  $n$ elements.

  \subsection{Finite Bags in Sets}
  In a first attempt, we define
  \begin{align*}
    \FMSet X
      \DefEq{}
      \sum\Var{sz} : ℕ\Where
        (\Fin \Var{sz} \to X) \SetQuot \mathrel{\sim_\Var{sz}},
  \end{align*}
  where $(\mathunderscore\sim_{\Var{sz}}\mathunderscore)$ relates
  $v, w : \Fin \Var{sz} → X$ iff there merely exists a permutation $\sigma$
  such that $v \circ \sigma = w$.
  We show that this definition enjoys a few desirable properties.
  First and foremost, it is a set regardless of the homotopy level of $X$.
  Secondly, it is invariant under set-truncation, i.e.
  $\FMSet \SetTrunc{X} ≃ \FMSet X$.
  In this sense, $\FMSet$ behaves like a functor of sets,
  which poses a problem when attempting to construct a final coalgebra from the
  above mentioned chain:
  Since $(\mathunderscore\sim_{\Var{sz}\mathunderscore})$ encodes permutations
  of multisets as property instead of data,
  proving preservation of limits has to perform propositional induction on
  countably many terms at once.
  More precisely, we prove the following theorem internally to the system:

  \begin{theorem}
    Injectivity of the limit-preservation map
    \[
        \operatorname{pres}_{\Lim (\Chain \M)}\colon
            \M (\Lim (\Chain \M))
            \to
            \Lim (\Shift(\Chain\M))
    \]
    implies the \emph{lesser limited principle of omniscience}, \LLPO.
  \end{theorem}
  \todo[inline]{%
    We prove this for the HIT-definition,
    but I state it here for the $\N$-indexed definition.
    Either say both are equivalent or prove it for the other definition.
  }

  \LLPO{} is a weaker version of the law of the excluded middle, and not
  provable from intuitionistic axioms alone.
  It states that for any sequence of boolean values $a_0, a_1, a_2, \ldots$
  that is $\mathsf{true}$ at most once, one can decide whether $a_k$ is
  $\mathsf{false}$ for all even $k$, respectively $\mathsf{false}$ for all odd
  $k$.

  \subsection{Finite Bags in Groupoids}

  To remedy the above situation, we give a second type of definition.
  This time, the bag functor will treat identifications of bags as data,
  hereby forcing each type of bags to have the homotopy level of at least a
  groupoid.
  We argue that this is the correct perspective on bags in HoTT, and substantiate the claim by the following theorem:

  \begin{theorem}
    The chain
    \[
      1 \xleftarrow{!} {\FMGpd 1}
        \xleftarrow{\FMGpd(!)} {\FMGpd^2 1}
        \xleftarrow{\FMGpd^2(!)} {\FMGpd^3 1}
        \xleftarrow{\hphantom{\FMGpd^3(!)}}
        \cdots
    \]
    stabilizes in $\omega$ many steps and yields an equivalence of groupoids
    $\FMGpd V_\omega ≃ V_\omega$.
  \end{theorem}

\bibliographystyle{splncs04}
\bibliography{Multiset}
\end{document}
