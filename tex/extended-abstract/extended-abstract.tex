\documentclass{easychair}

\usepackage{multiset}
\usepackage{todonotes}

\usepackage{thmtools}
\usepackage{bussproofs}
\usepackage{csquotes}

\declaretheorem{theorem}
\declaretheorem{definition}

\begin{document}
  \title{Final Coalgebra of the Finite Bag Functor}
  \titlerunning{Final Coalgebra of the Finite Bag Functor}
  \author{%
      Philipp Joram \and
      Niccol{\`o} Veltri
  }
  %
  \authorrunning{P. Joram and N. Veltri}
  \institute{Department of Software Science, Tallinn University of Technology, Estonia}

  \maketitle

  The powerset and multiset (or bag) functor, delivering the set of
  subsets (resp. multisubsets) of a given set, are fundamental
  mathematical tools in the behavioural analysis of nondeterministic
  systems. Such systems can be described as coalgebras $c : S \to
  F\, S$, with $F$ being either powerset or bag functor, associating
  to each state $x : S$ the collection $c \, x$ of reachable
  states. When $F$ is the bag functor, a state $x$ can reach
  another state $y$ in multiple ways, specified by the multiplicity of
  $y$ in $c\,x$. In practice one often studies systems with finite reachable
  states and employs finite variants of the powerset and bag functor.
  The finite bag functor takes a set $X$ to the set of finite
  bags of elements of $X$.  In contrast to the finite powerset
  functor, these collections distinguish multiple occurrences of
  identical elements, e.g.\@ $\{1, 2, 1\}$ is a bag containing an element
  1 with multiplicity 2.
  
  The behavior of a finitely nondeterministic system starting from a
  given initial state $x : S$ is fully captured by the final coalgebra
  of the collection functor $F$, whose elements are non-wellfounded
  trees obtained by iteratively running the coalgebra $c$ on $x$. In
  recent work \cite{Veltri2021}, we considered various constructions
  of the final coalgebra of the finite powerset functor in the setting
  of Homotopy Type Theory (HoTT). In this work, we explore possible
  definitions of the finite bag functor in HoTT and investigate
  whether they admit a final coalgebra.

  In classical set theory, the final coalgebra of the finite bag
  functor $F$ is obtained as the limit of the
  $\omega^{\operatorname{\mathbf{op}}}$-chain
  \cite[{Ch. 3.3.13}]{Adamek2021}
  \begin{equation}\label{eq:chain}
    1 \xleftarrow{!} {F 1}
      \xleftarrow{F(!)} {F^2 1}
      \xleftarrow{F^2(!)} {F^3 1}
      \xleftarrow{F^3(!)}
      \cdots
  \end{equation}
  where $F^n 1$ is the $n$-th iteration of the functor $F$ on the
  singleton set 1, and $F^n(!)$ is the $n$-th iteration of the
  functorial action of $F$ on the unique function $!$ targeting the set
  $1$. In a constructive environment such as HoTT, it is not
  immediately obvious whether this construction still produces the
  final coalgebra or not.

  We formalized our work in \emph{Cubical Agda} \cite{Vezzosi2019},
  which is a particular implementation of Homotopy Type Theory
  with support for the univalence principle (\enquote{equivalence of types is equivalent to equality of types}, which is a provable theorem in Cubical Agda)
  and higher inductive types.
  A distinguishing feature of Homotopy Type Theory is that the identity type on a type $A$
  is no longer inductively defined as in Martin-L{\"o}f Type Theory (MLTT).
  In particular, for proofs of identification $p, q : a =_A b$, the iterated identity type
  $p =_{a =_A b} q$ might be inhabited by more than one term, which in turn might have non-trivial
  identity type and so on.
  One says that $A$ has \emph{homotopy level} $n$ if the $n$-th iterated identity type is trivial.
  We explicitly name the first instances of this hierarchy, and say that $A$ is:
  \begin{itemize}
    \item ($n = 1$) a \emph{proposition}, if
      $\IsProp A \DefEq ∀ (a , b : A)\Where a = b$ is inhabited,
    \item ($n = 2$) a \emph{set}, if
      $\IsSet A \DefEq ∀ (a, b : A)\Where \IsProp(a = b)$ is inhabited,
    \item ($n = 3$) a \emph{groupoid}, if
      $\IsGpd A \DefEq ∀ (a, b : A)\Where \IsSet(a = b)$ is inhabited.
  \end{itemize}
  We stress that, when mentioning \enquote{sets} or \enquote{groupoids}, we always refer to these definitions.

  In ordinary MLTT, the principle of uniqueness of identity proofs (UIP) implies that all types are sets.
  This is not true for arbitrary types in Homotopy Type Theory.
  Using \emph{higher inductive types}, one can obtain a set from any type:
  \begin{definition}
    For any type $A$, the set-truncation $\SetTrunc{A}$ is the type inductively defined by
    \begin{center}
      \hspace*{\fill}
        \AxiomC{$a : A$}
        \UnaryInfC{
          $\SetTruncCon[a] : \SetTrunc{A}$
        }
        \DisplayProof
      \hfill
        \AxiomC{$x, y : \SetTrunc{A}$}
        \AxiomC{$p, q : x = y$}
        \RightLabel{$\operatorname{squash}_2$}
        \BinaryInfC{$p = q$}
        \DisplayProof
      \hspace*{\fill}
    \end{center}
  \end{definition}
  This type is a \enquote{higher} inductive type since the rule $\operatorname{squash}_2$ produces an identification $p = q$ instead of an element of $\SetTrunc{A}$.
  By definition, $\operatorname{squash}_2 : \IsSet \SetTrunc{A}$.
  Similarly, higher inductive types allow us to model quotient types,
  in particular set-quotients:

  \begin{definition}
    Given $A : \Type$ and a binary relation $(\sim) : A \to A \to \Type$,
    the \emph{set-quotient} of $A$ by $(\sim)$ is given by rules
    \begin{center}
      \hspace*{\fill}
        \AxiomC{$a : A$}
        \UnaryInfC{
          $\SetQuotCon[a] : \SetQuot[A][\sim]$
        }
        \DisplayProof
      \hfill
        \AxiomC{$a, b : A$}
        \AxiomC{$r : a \sim b$}
        \RightLabel{$\operatorname{eq/}_{\!2}$}
        \BinaryInfC{$\SetQuotCon[a] = \SetQuotCon[b]$}
        \DisplayProof
      \hfill
        \AxiomC{$x, y : \SetQuot[A][\sim]$}
        \AxiomC{$p, q : x = y$}
        \RightLabel{$\operatorname{squash/}_{\!2}$}
        \BinaryInfC{$p = q$}
        \DisplayProof
      \hspace*{\fill}
    \end{center}
    and forms a set by $\operatorname{squash/}_{\!2}$.
  \end{definition}
  Similarly, the \emph{propositional truncation} of $A$ is the higher inductive type $\PropTrunc{A}$
  with a constructor $\PropTruncCon : A → \PropTrunc{A}$
  and a rule $\operatorname{squash}_1 : \IsProp \PropTrunc{A}$.

  \subsection*{Finite Bags in Sets}
  In a first attempt, we define
  \begin{align*}
    \FMSet X
      \DefEq{}
      \sum\Var{k} : ℕ\Where
        \SetQuot[(\Fin \Var{k} \to X)][\sim_\Var{sz}],
  \end{align*}
  where $\Fin \Var{k}$ is the type of numbers $<k$ and $(\SymmetricAction{\Var{sz}}[][])$ relates
  $v, w : \Fin \Var{sz} → X$ iff there merely exists a permutation $\sigma$
  such that $v \circ \sigma = w$. So a finite bag consists of a number $k$ (its size) and an equivalence class of $k$-elements of $X$, where the relation $(\SymmetricAction{\Var{sz}}[][])$ expresses that the order of the elements does not matter. 
  $\FMSet{X}$ is a set, regardless of the homotopy level of $X$.
  Additionally, $\FMSet$ is invariant under set-truncation, i.e.
  $\FMSet \SetTrunc{X} ≃ \FMSet X$.
  The type of finite bags can equivalently be defined as the free commutative monoid on $X$, which can be directly expressed as a higher inductive type \cite{Choudhury2021}.

  Trying to construct the final coalgebra of $\FMSet$ as the limit of
  the chain (\ref{eq:chain}) (as traditionally done in a classical
  metatheory) is problematic. The first step in the construction
  would be showing that $\FMSet$ preserves $\omega$-limits. In the
  case of the chain (\ref{eq:chain}) can be reduced to show that the
  map $ \operatorname{pres}\colon \FMSet (\lim_{n < \omega} \FMSet^n
  1) \to \lim_{n < \omega} (\FMSet^{n+1} 1)$, defined via the
  universal property of the limit, is an equivalence of types. This is not the case in HoTT, since the latter is an inherently non-constructive principle:

  \begin{theorem}\label{thm:InjPresImpliesLLPO}
    The function 
        $\operatorname{pres}\colon
            \FMSet (\lim_{n < \omega} \FMSet^n 1)
            \to
            \lim_{n < \omega} (\FMSet^{n+1} 1)$
    is surjective,
    but its injectivity is equivalent to the \emph{lesser limited principle of omniscience}, \LLPO.
  \end{theorem}
  \LLPO{} \cite[{Ch.\@ 1}]{Bridges1987} is a weak version of the law
  of the excluded middle, and it is not provable from intuitionistic
  axioms alone.  It states that, given an infinite stream of boolean
  values that yields $\True$ in at most one position, one can decide
  whether all even or all odd positions are $\False$.

  The non-constructive nature of the injectivity of
  $\operatorname{pres}$ can be attributed to the fact that the
  relation $(\SymmetricAction{\Var{sz}}[][])$ encodes permutations of
  multisets as property instead of data, and this makes it impossible
  to recover information about all terms in the chain.

  We conclude from Theorem \ref{thm:InjPresImpliesLLPO} that the
  classical construction of the final coalgebra for $\FMSet$ cannot be
  replicated in our constructive setting without assumption of
  classical principles. This is analogous to the case of the finite
  powerset functor, which we know to be suffering from similar issues \cite{Veltri2021}.

  \subsection*{Finite Bags in Groupoids}

  To remedy the situation, we introduce a bag functor that treats identifications of bags as data.
  Define the type of \emph{(Bishop) finite sets} as
  \[
    \FinSet \DefEq
      \sum Y : \Type\Where
        \sum k : \N\Where
        \PropTrunc{Y ≃ \Fin k},
  \]
  i.e.\@ the type of all types merely equivalent to some $\Fin k$.
  While in this abstract we suppress any size-related issues (all the definitions and theorem given so far can be made universe-polymorphic), they play a crucial r\^{o}le in the formalization.
  Notice that $\FinSet$ is a large type compared to the types it ranges over,
  but it admits a small axiomatization as a higher inductive type $\Bij$, originally introduced in \cite{Finster2021}.
  Its introduction rules are the following (plus one stating that $\Bij$ is a groupoid):
  \begin{center}
    \small
    \hspace*{\fill}
      \AxiomC{$n : \N$}
      \UnaryInfC{
        $\Obj n : \Bij$
      }
      \DisplayProof
    \hfill
      \AxiomC{$m, n : \N$}
      \AxiomC{$\alpha : \Fin m ≃ \Fin n$}
      \RightLabel{$\Hom$}
      \BinaryInfC{$\Obj m = \Obj n$}
      \DisplayProof
    \hfill
      \AxiomC{$n : \N$}
      \UnaryInfC{$\Hom (\operatorname{id}_{\Fin n}) = \Refl$}
      \DisplayProof
    \hspace*{\fill}
    \\[1em]
    \hspace*{\fill}
      \AxiomC{$m, n, o : \N$}
      \AxiomC{$\alpha : \Fin m ≃ \Fin n$}
      \AxiomC{$\beta : \Fin n ≃ \Fin o$}
      \TrinaryInfC{$\Hom(\beta \circ \alpha) = \Hom \alpha \bullet \Hom \beta$}
      \DisplayProof
    \hspace*{\fill}
  \end{center}
  Here, $\operatorname{id}$ is the identity-equivalence, $(\circ)$ composition of equivalences,
  $\Refl : \Obj n = \Obj n$ a reflexivity proof of identity and $(\bullet)$ is transitivity of $(=)$.

  With this in mind, we define, for any type $X$, the type
  \[
    \FMGpd X \DefEq
      \sum x : \Bij\Where \langle x \rangle → X,
  \]
  where $\langle\_\rangle : \Bij → \Type$ obtains the type corresponding to $x : \Bij$ from the
  equivalence $\Bij ≃ \FinSet$.
  Since $\FinSet$ is a groupoid, each $\FMGpd X$ has a homotopy level of at least that of a groupoid.
  By identifying the higher structure of $\FMGpd$, we recover $\FMSet$:
  \begin{theorem}
    For any type $X$, there is an equivalence $\SetTrunc{\FMGpd X} ≃ \FMSet X$.
  \end{theorem}

  
  Following \cite{Kock2012}, we argue that this is the correct perspective on bags in Homotopy Type Theory,
  and substantiate the claim by the following unproblematic construction of the final coalgebra of $\FMGpd$:

  \begin{theorem}\label{thm:FMGpdLim}
    Let $L_{\FMGpd} \DefEq \lim_{n < \omega} \FMGpd^n 1$.
    The limit-preservation map $\operatorname{pres}$ is an equivalence of groupoids.
    In particular, the limit $L_{\FMGpd}$ is a fixpoint of $\FMGpd$ and its final coalgebra.
  \end{theorem}
  This theorem is a direct consequence of a general result by Ahrens
  et al. \cite{Ahrens2015}, since $\FMGpd$ is a polynomial functor in
  groupoids, and all polynomial functors admit a final coalgebra in
  HoTT, independently of the their homotopy level.  The existence of
  $L_{\FMGpd}$ requires $\FMGpd$ to be defined in terms of $\Bij$,
  otherwise it is not typeable in a universe of fixed size.

  One might wonder whether the final coalgebra in groupoids can be used to define a final coalgebra also in sets.
  We are able to show that the set-truncation of the groupoid-limit \SetTrunc{L_{\FMGpd}} is a fixpoint of $\FMSet$, i.e. $\FMSet \SetTrunc{L_{\FMGpd}} ≃ \SetTrunc{L_{\FMGpd}}$. But further showing that this is the final coalgebra of the set-based bag functor $\FMSet$ seems to require the full axiom of choice.

  \section*{Acknowledgments}
  This work was supported by the Estonian Research Council grant PSG749.

\bibliographystyle{splncs04}
\bibliography{Multiset}
\end{document}
