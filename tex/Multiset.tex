% This is samplepaper.tex, a sample chapter demonstrating the
% LLNCS macro package for Springer Computer Science proceedings;
% Version 2.21 of 2022/01/12
%
\documentclass[runningheads]{llncs}
%
\usepackage[T1]{fontenc}
\usepackage{multiset}

\usepackage{bussproofs}

\usepackage{textcomp}

\usepackage{todonotes}

% Used for displaying a sample figure. If possible, figure files should
% be included in EPS format.
%
% If you use the hyperref package, please uncomment the following two lines
% to display URLs in blue roman font according to Springer's eBook style:
%\usepackage{color}
%\renewcommand\UrlFont{\color{blue}\rmfamily}
%

\usepackage{latex/agda}

\usepackage{fancyvrb}
\DefineVerbatimEnvironment
  {code}{Verbatim}
  {}
\begin{document}

\title{Final Coalgebra of the Finite Bag Functor}
\author{%
    Philipp Joram\inst{1}\orcidID{0000-0002-0448-7907} \and
    Niccolò Veltri\inst{1}\orcidID{0000-0002-7230-3436}%
}
%
\authorrunning{P. Joram et al.}
\institute{Department of Software Science, Tallinn University of Technology, Estonia}

\maketitle              % typeset the header of the contribution

\begin{abstract}
The abstract should briefly summarize the contents of the paper in
150--250 words.

\keywords{First keyword  \and Second keyword \and Another keyword.}
\end{abstract}

\section{Introduction}

\subsection{Notation}
\begin{itemize}
  \item Definitional equality: $\DefEq$
  \item propositional equality (paths): $=$
  \item Binders: always in the form of $\mathsf{Q} (x : X)\Where \langle\mathsf{expr}\rangle$,
    i.e. $\sum (n : \N)\Where n * n = 2$.
    Binders extend as far right as possible.
  \item $\exists$ for \emph{mere existence}, i.e. $\PropTrunc{\sum (\mathunderscore)\Where X}$.
  \item dependent paths: $\PathP{} (\lambda i\Where \mathunderscore)\, \mathunderscore\, \mathunderscore$
\end{itemize}

\section{Type Theory and Cubical Agda}

\section{The Finite Bag Functor}

The action of the bag functor on a type $X$ can be encoded as higher inductive
types in various ways, two of which we present here.
The first is as the type of lists set-quotiented by an \emph{up to permutation}-relation,
the second being the free commutative monoid on $X$.


\subsection{As a HIT}

Given a type $X$, the free commutative monoid $(\M X, \Union, \Empty)$ is
defined as the higher inductive type induced by the following rules:
\begin{itemize}
  \item Point constructors:
    \begin{center}
      \hspace*{\fill}
        \AxiomC{$\vphantom{X}$}
        \UnaryInfC{$\Empty : \M{X}$}
        \DisplayProof
      \hfill
        \AxiomC{$x : X$}
        \UnaryInfC{$\Singl x : \M X$}
        \DisplayProof
      \hfill
        \hspace{10pt}
        \AxiomC{$xs, ys : \M X$}
        \UnaryInfC{$\Var{xs} \Union \Var{ys} : \M X$}
        \DisplayProof
      \hspace*{\fill}
    \end{center}
  \item Monoid axioms and commutativity:
    \begin{center}
      \hspace*{\fill}
        \AxiomC{$\Var{xs} : \M X$}
        \UnaryInfC{$\mathsf{unit} : \Empty \Union \Var{xs} = \Var{xs}$}
        \DisplayProof
      \hfill
        \AxiomC{$\Var{xs}, \Var{ys}, \Var{xs} : \M X$}
        \UnaryInfC{
          $\mathsf{assoc} : \Var{xs}\Union(\Var{ys}\Union\Var{zs}) = (\Var{xs}\Union\Var{ys})\Union\Var{zs}$
        }
        \DisplayProof
      \hspace*{\fill}
      \\[6pt]
      \hspace*{\fill}
        \AxiomC{$\Var{xs}, \Var{ys} : \M X$}
        \UnaryInfC{
          $\mathsf{comm} : \Var{xs}\Union\Var{ys} = \Var{ys}\Union\Var{xs}$
        }
        \DisplayProof
      \hspace*{\fill}
    \end{center}
  \item set truncation:
    \begin{center}
      \hspace*{\fill}
        \AxiomC{$\Var{xs}, \Var{ys} : \M X$}
        \AxiomC{$p, q : \Var{xs} = \Var{ys}$}
        \BinaryInfC{
          $\mathsf{trunc} : p = q$
        }
        \DisplayProof
      \hspace*{\fill}
    \end{center}
\end{itemize}

\subsection{As a Set Quotient}

\begin{definition}
  The type of finite multisets over a type $X$ is the type
  \begin{align*}
    \FMSet X
      \mathbin{=_{\mathrm{df}}}
      \sum\,\Var{sz} : ℕ\Where
        (\Fin \Var{sz} \to X) \mathbin{/_{\!2}} \mathrel{\sim},
  \end{align*}
  where $\sim$ is the proposition-valued relation defined as
  \begin{align*}
    \textunderscore\sim\textunderscore &: ∀ \{n\}\Where (v\, w : \Fin n → X) → \Type \\
    v \sim w &\mathrel{=_{\mathsf{df}}}
      ∃ \sigma : \Fin n ≃ \Fin n \Where
        \PathP{} (\lambda i\Where \operatorname{ua}(\sigma)\, i → X)\, v\, w
  \end{align*}
\end{definition}

\subsubsection{Finite Choice for Sets}

\begin{lemma}
  Set truncation distributes over finite families of types.
  For any $n : ℕ$ and type family $Y : \Fin n \to \Type$,
  there is an equivalence
  \[
    \Op{box} :
    ((k : \Fin n) → \SetTrunc{\mathop{Y\/} k})
    ≃
    {\SetTrunc{(k : \Fin n) → \mathop{Y\/} k}}
  \]
\end{lemma}

\begin{definition}
  The principle of finite choice can be defined in terms of $\Op{box}$:
  \begin{align*}
    \Op{elim}_{\Op{fin}} &: \forall n\Where \\
      &\to \{B : (\Fin n \to \SetTrunc{X}) \to \Type\} \\
      &\to (\forall v\Where \operatorname{isSet} (B\, v)) \\
      &\to ((\mathsf{choice} : \Fin n \to X) \to B\, (\SetTruncCon{} \circ \mathsf{choice})) \\
      &\to (v : \Fin n \to \SetTrunc{X}) \to B\, v)
  \end{align*}
  It is defined by applying $\mathsf{choice}$ to the term obtained from
  set-truncation elimination on $\operatorname{box} v$.
  It comes with a computational rule
  \begin{align*}
    \Op{elim}^\beta_{\Op{fin}} : (v : \Fin n → X)
      → \Op{elim}_{\Op{fin}} (\SetTruncCon{} \circ v) = \operatorname{\mathsf{choice}} v
  \end{align*}
\end{definition}

\begin{theorem}\label{thm:FMSetSetTruncInvariant}
  Finite multisets are invariant under set-truncation:
  \begin{equation}
    \FMSet \SetTrunc{X} ≃ \FMSet X
  \end{equation}
\end{theorem}
\begin{proof}
  \begin{enumerate}
    \item Inverse is given by $\operatorname{map} \SetTruncCon{} : \FMSet X → \FMSet \SetTrunc{X}$.
    \item Define the inverse by using
      \[
        \Op{requot} : ∀ \{n\}\Where
          (\Fin n → \SetTrunc{X})
          → (\Fin n → X) \SetQuot{} {\sim_n},
      \]
      defined via $\Op{elim}_{\Op{fin}}$.
  \end{enumerate}
\end{proof}

\subsection{As a Groupoid}
Following \cite{Kock2012}, we argue the correct perspective on bags in Homotopy Type Theory
is to define them as groupoids instead of sets.
The rationale is that identifications of bags are permutations, and these should inherently be treated as \emph{data}.
Instead of viewing bags as quotients of lists, thereby \enquote{forgetting} about the permutations,
we define a type of lists with \enquote{more identifications}.
Since all constructions based on this type have to be homotopy coherent,
they will automatically respect these extra data,
making them invariant under permutation for free.

We start by defining a type family $\Bag : \Type → \Type$,
and substantiate the above claims by first showing its set-truncation is $\FMSet$ (\cref{thm:FMSetOfFMGpdTrunc}),
and in \cref{ssec:FMGpdLim} by constructing its final coalgebra (\cref{thm:FMGpdLim}).


\begin{itemize}
    \item Compare large/small definition.
\end{itemize}
\begin{definition}
  A small (\enquote{sm{\aa}l}) type:
  \begin{align*}
    \Bag X
      \mathbin{=_{\mathrm{df}}}
      \sum\,\Var{x} : \operatorname{\mathsf{Bij}}\Where
        \langle x \rangle \to X,
  \end{align*}
\end{definition}
\begin{definition}
  A large type, $\Tote : \Type_\ell → \Type_{\ell + 1}$:
  \begin{align*}
    \Tote X
      \mathbin{=_{\mathrm{df}}}
      \sum\,\Var{B} : \operatorname{FinSet}\Where B \to X,
  \end{align*}
\end{definition}

\begin{proposition}
  If $X$ is a groupoid, then $\Bag X$ is a groupoid.
\end{proposition}
\begin{proof}
  By assumption $X$ is a groupoid,
  so the function type $\langle x \rangle → X$ is a groupoid
  for any $x : \Bij$.
  By construction $\Bij$ is a groupoid, therefore the entire $\Sigma$-type is a groupoid.
\end{proof}

\paragraph{Comparing $\Op{List}$ and $\Bag$}\todo{Fit this comparison in somewhere}
Note that $\Bag$ is somewhat analogous to $\Op{List}$:
Given a type $A$ and a family $B$ over it, they both match ${\sum (a : A)\Where B(a) → X}$:
we have $A = ℕ$ and $B = \Fin$ for $\Op{List}$,
and $A = \Bij$ and $B = \langle\_\rangle$ for $\Bag$.
In particular, $\SetTrunc{\Bij} ≃ ℕ$.

In general, $\FMSet X$ is at least a groupoid.

\begin{theorem}\label{thm:FMSetOfFMGpdTrunc}
  Set-truncating a bag yields a finite multiset.
  For any type $X$,
  \[
    \SetTrunc{\Bag X} ≃ \FMSet X
  \]
\end{theorem}
\begin{proof}
  We prove the result for $\Tote$, and later transport it to $\Bag$.
  The proof proceeds by constructing an explicit isomorphism $\SetTrunc{\Tote X} \cong \FMSet X$.
  To construct the map $\FMSet X → \SetTrunc{\Tote X}$, by the induction principle for
  $\FMSet$,\todo{Define this earlier!} it is enough to give a function
  $f : \forall \{\Var{sz}\}\Where (\Fin \Var{sz} → X) → \SetTrunc{\Tote X}$
  and to show that it respects $(\sim)$.
  Let $f(v) \DefEq \SetTruncCon[(\Fin \Var{sz} , v)]$, since $\Fin \Var{sz}$ is obviously a finite set.
  To prove that $v \sim w$ implies $f(v) = f(w)$, we note that the conclusion is a proposition,
  thus we can assume a permutation $v \circ \sigma = w$ to construct \ldots
  \todo[inline]{%
    CONTINUE HERE.
  }
\end{proof}

\subsection{As lists up to permutation}

\begin{theorem}
  For any $X$, the types $\FMSet X$, $\M X$ and $\operatorname{PList} X$ are equivalent.
\end{theorem}

\section{The Final Coalgebra}

Describe limits in general.
Maybe compare final coalgebras and corecursive algebras which are fixpoints.

\subsection{Final Coalgebras as an $\omega$-Limit in Set}

\todo[inline]{%
    Surjectivity works (under the assumption of countable(?) choice).
}

  \begin{theorem}\label{thm:InjPresImpliesLLPO}
    The function 
        $\operatorname{pres}\colon
            \FMSet (\lim_{n < \omega} \FMSet^n 1)
            \to
            \lim_{n < \omega} (\FMSet^{n+1} 1)$
    is surjective,
    but its injectivity implies the \emph{lesser limited principle of omniscience}, \LLPO.
  \end{theorem}

\todo[inline]{%
    Think about the converse statement,
    does {\LLPO} already imply injectivity?
}

\subsection{Final Coalgebras as an $\omega$-Limit in Groupoids}\label{ssec:FMGpdLim}

\begin{theorem}\label{thm:FMGpdLim}
  Let $L_{\Bag} \DefEq \lim_{n < \omega} \Bag^n 1$.
  The limit-preservation map $\operatorname{pres}$ is an equivalence of groupoids.
  In particular, the limit $L_{\Bag}$ is a fixpoint of $\Bag$ and its final coalgebra.
\end{theorem}

\subsection{Truncating the Groupoid Construction}

Compare the set-truncation of the groupoid construction
and the set-level definition.

\begin{theorem}\label{thm:FMSetFixpointOfTrunc}
  The set-truncation of $\Lim{\Bag}$ is a fixpoint of $\FMSet$, i.e.\@
  there is an equivalence $\FMSet \SetTrunc{\Lim{\Bag}} ≃ \SetTrunc{\Lim{\Bag}}$.
\end{theorem}
\begin{proof}
  The equivalence is obtained from the composition
  \begin{align}
    \FMSet \SetTrunc{\Lim{\Bag}}
      &≃ \FMSet \Lim{\Bag}          \Step{thm:FMSetFixpointOfTrunc:step1} \\
      &≃ \SetTrunc{\Bag \Lim{\Bag}} \Step{thm:FMSetFixpointOfTrunc:step2} \\
      &≃ \SetTrunc{\Lim{\Bag}}      \Step{thm:FMSetFixpointOfTrunc:step3}
  \end{align}
  where \eqref{thm:FMSetFixpointOfTrunc:step1} is invariance of $\FMSet$ under set-truncation
  (\cref{thm:FMSetSetTruncInvariant}),
  \cref{thm:FMSetFixpointOfTrunc:step2} is \cref{thm:FMSetOfFMGpdTrunc}
  and \eqref{thm:FMSetFixpointOfTrunc:step3} follows since $\Lim{\Bag}$ is a limit (\cref{thm:FMGpdLim}).
\end{proof}

Note that the above fixpoint is not necessarily the \emph{largest} fixpoint.
\todo[inline,caption={Choice for largest fixpoint}]{%
  \emph{(Unproven)}
  Only assuming the (full) axiom of choice this is the case, i.e.\@ yields a final coalgebra.
  }

\section{Discussion: Alternatives}
\subsection{Outlook: Generalization to Analytic Functors}

\begin{itemize}
    \item Hint at a definition of analytic functors using
        the above definitions (pick a subgroupoid of Bij etc.)
    \item Question:
        Does this definition \emph{weakly preserve pullbacks}?
        This would be the classical definition.
\end{itemize}

\subsection{Using Coinductive Types?}

\begin{itemize}
    \item It is not clear that Agda's coinductive types
        interacts well with HITs.
    \item This might not work with the groupoid-definition:
        Either we use HITs (possibly inconsistent, for the small Bij),
        or the construction ends up being too large.
    \item
        Another approach: Quotient the (entire) final coalgebra of lists.
\end{itemize}

\section{Conclusions}

\subsubsection{Acknowledgements}

\nocite{*}

\bibliographystyle{splncs04}
\bibliography{Multiset}
\end{document}
