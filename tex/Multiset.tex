\documentclass[runningheads]{llncs}

\usepackage[debug]{multiset}

\usepackage{bussproofs}

\usepackage{todonotes}

\begin{document}

\title{Final Coalgebra of the Finite Bag Functor}
\author{%
    Philipp Joram\orcidID{0000-0002-0448-7907} \and
    Niccolò Veltri\orcidID{0000-0002-7230-3436}%
}
%
\authorrunning{P. Joram et al.}
\institute{Department of Software Science, Tallinn University of Technology, Estonia}

\maketitle              % typeset the header of the contribution

\begin{abstract}
The abstract should briefly summarize the contents of the paper in
150--250 words.

\keywords{First keyword  \and Second keyword \and Another keyword.}
\end{abstract}

\section{Introduction}

\subsection{Notation}
\begin{itemize}
  \item Definitional equality: $\DefEq$
  \item propositional equality (paths): $=$
  \item Binders: always in the form of $\mathsf{Q} (x : X)\Where \langle\mathsf{expr}\rangle$,
    i.e. $\sum (n : \N)\Where n * n = 2$.
    Binders extend as far right as possible.
  \item $\exists$ for \emph{mere existence}, i.e. $\PropTrunc{\sum (\mathunderscore)\Where X}$.
  \item dependent paths: $\PathP{} (\lambda i\Where \mathunderscore)\, \mathunderscore\, \mathunderscore$
  \item eliminators and co.:
    For any $F : \Type → \Type$, write
    $\Elim{F}$, $\Rec{F}$ and $\Map{F}$
    for the (dependent) eliminator, recursor and action on maps.
\end{itemize}

\section{Type Theory and Cubical Agda}

\section{The Finite Bag Functor}

The action of the bag functor on a type $X$ can be encoded as higher inductive
types in various ways, two of which we present here.
The first is as the type of lists set-quotiented by an \emph{up to permutation}-relation,
the second being the free commutative monoid on $X$.


\subsection{As a HIT}

Given a type $X$, the free commutative monoid $(\M X, \Union, \Empty)$ is
defined as the higher inductive type induced by the following rules:
\begin{itemize}
  \item Point constructors:
    \begin{center}
      \hspace*{\fill}
        \AxiomC{$\vphantom{X}$}
        \UnaryInfC{$\Empty : \M{X}$}
        \DisplayProof
      \hfill
        \AxiomC{$x : X$}
        \UnaryInfC{$\Singl x : \M X$}
        \DisplayProof
      \hfill
        \hspace{10pt}
        \AxiomC{$xs, ys : \M X$}
        \UnaryInfC{$\Var{xs} \Union \Var{ys} : \M X$}
        \DisplayProof
      \hspace*{\fill}
    \end{center}
  \item Monoid axioms and commutativity:
    \begin{center}
      \hspace*{\fill}
        \AxiomC{$\Var{xs} : \M X$}
        \UnaryInfC{$\mathsf{unit} : \Empty \Union \Var{xs} = \Var{xs}$}
        \DisplayProof
      \hfill
        \AxiomC{$\Var{xs}, \Var{ys}, \Var{xs} : \M X$}
        \UnaryInfC{
          $\mathsf{assoc} : \Var{xs}\Union(\Var{ys}\Union\Var{zs}) = (\Var{xs}\Union\Var{ys})\Union\Var{zs}$
        }
        \DisplayProof
      \hspace*{\fill}
      \\[6pt]
      \hspace*{\fill}
        \AxiomC{$\Var{xs}, \Var{ys} : \M X$}
        \UnaryInfC{
          $\mathsf{comm} : \Var{xs}\Union\Var{ys} = \Var{ys}\Union\Var{xs}$
        }
        \DisplayProof
      \hspace*{\fill}
    \end{center}
  \item set truncation:
    \begin{center}
      \hspace*{\fill}
        \AxiomC{$\Var{xs}, \Var{ys} : \M X$}
        \AxiomC{$p, q : \Var{xs} = \Var{ys}$}
        \BinaryInfC{
          $\mathsf{trunc} : p = q$
        }
        \DisplayProof
      \hspace*{\fill}
    \end{center}
\end{itemize}

\subsection{As a Set Quotient}

\begin{definition}
  The type of finite multisets over a type $X$ is the type
  \begin{align*}
    \FMSet X
      \mathbin{=_{\mathrm{df}}}
      \sum\,\Var{sz} : ℕ\Where
        (\Fin \Var{sz} \to X) \mathbin{/_{\!2}} \mathrel{\sim},
  \end{align*}
  where $\sim$ is the proposition-valued relation defined as
  \begin{align*}
    v \sim w &\mathrel{=_{\mathsf{df}}}
      ∃ \sigma : \Fin n ≃ \Fin n \Where
        \PathP{} (\lambda i\Where \operatorname{ua}(\sigma)\, i → X)\, v\, w
  \end{align*}
\end{definition}

\begin{definition}
  The recursion principle for $\FMSet X$ is the function
  \begin{align*}
    \Rec{\FMSet}
      &: \{A : \Type\} → \IsSet{A} \\
      &→ (\Op{as} : \{n : ℕ\} → (v : \Fin n → X) → A) \\
      &→ (∀ n\Where (v, w : \Fin n → X) → v \sim_n w → \Op{as} v = \Op{as} w) \\
      &→ \FMSet X → A
  \end{align*}
  defined from the recursion principle of set-quotients.
  Similarly, we define the induction principle $\Op{elim}_{\FMSet}$ for
  a dependent type family $B : \FMSet X → \Type$ of sets.
\end{definition}

\subsubsection{Finite Choice for Sets}

\begin{lemma}
  Set truncation distributes over finite families of types.
  For any $n : ℕ$ and type family $Y : \Fin n \to \Type$,
  there is an equivalence
  \[
    \Op{box} :
    ((k : \Fin n) → \SetTrunc{\mathop{Y\/} k})
    ≃
    {\SetTrunc{(k : \Fin n) → \mathop{Y\/} k}}
  \]
\end{lemma}
\begin{proof}
  We sketch a proof for a constant type family $Y = (\lambda \Blank \Where X)$.
  The dependent case is analogous.
  The function underlying the equivalence is defined by induction on $n$.
  For $n = 0$ we have $\Fin 0 ≃ \bot$,
  so $\Op{box} \DefEq (\lambda \Blank\Where \SetTruncCon[\Elim{\bot}])$.
  In the inductive step, we lift the \enquote{cons} operation
  $
    (\Cons) : X → (\Fin n → X) → (\Fin{} (1 + n) → X)
  $
  on vectors to the set-truncation.
  By induction on $n$, we see that
  \begin{align*}
    &\Op{unbox} : \SetTrunc{\Fin n → X} → \Fin k → \SetTrunc{X} \\
    &\Op{unbox} \bar{v}\, k \DefEq \Map{\SetTrunc{\Blank}} (\lambda v\Where v\, k)\, \bar{v}
  \end{align*}
  is a two-sided inverse of $\Op{box}$.
  The proof uses the fact that $\Map{\SetTrunc{\Blank}}$ is functorial.
\end{proof}

\begin{definition}
  The principle of finite choice can be defined in terms of $\Op{box}$:
  \begin{align*}
    \FinElim &: \forall n\Where \\
      &\to \{B : (\Fin n \to \SetTrunc{X}) \to \Type\} \\
      &\to (\forall v\Where \IsSet (B\, v)) \\
      &\to ((\mathsf{choice} : \Fin n \to X) \to B\, (\SetTruncCon{} \circ \mathsf{choice})) \\
      &\to (v : \Fin n \to \SetTrunc{X}) \to B\, v)
  \end{align*}
  It is defined by applying $\mathsf{choice}$ to the term obtained from
  set-truncation elimination on $\operatorname{box} v$.
  It comes with a computational rule
  \begin{align*}
    \FinElimComp : (v : \Fin n → X)
      → \FinElim (\SetTruncCon{} \circ v) = \operatorname{\mathsf{choice}} v
  \end{align*}
\end{definition}

\begin{theorem}\label{thm:FMSetSetTruncInvariant}
  Finite multisets are invariant under set-truncation:
  \begin{equation}
    \FMSet \SetTrunc{X} ≃ \FMSet X
  \end{equation}
\end{theorem}
\begin{proof}
  We obtain the equivalence from an isomorphism.
  The inverse map is set-truncation of the members of the finite multiset.
  In the forward direction, we use $\FinElim$ to define a function
  \[
    \Op{requot} : ∀ \{n\}\Where
    (\Fin n → \SetTrunc{X})
    → \SetQuot[(\Fin n → X)][\sim_n]
  \]
  that turns set-truncation into a set-quotient and prove that it preserves $(\sim_n)$.
  By recursion on $\FMSet \SetTrunc{X}$, this enough to obtain $\FMSet \SetTrunc{X} → \FMSet X$.
  That these maps are mutual inverses follows from $\FinElimComp$.
\end{proof}

\subsection{As a Groupoid}
Following \cite{Kock2012}, we argue the correct perspective on bags in Homotopy Type Theory
is to define them as groupoids instead of sets.
The rationale is that identifications of bags are permutations, and these should inherently be treated as \emph{data}.
Instead of viewing bags as quotients of lists, thereby \enquote{forgetting} about the permutations,
we define a type of lists with \enquote{more identifications}.
Since all constructions based on this type have to be homotopy coherent,
they will automatically respect these extra data,
making them invariant under permutation for free.
We define a type family $\Bag : \Type → \Type$,
and substantiate these claims by first showing its set-truncation is $\FMSet$ (\cref{thm:FMSetOfFMGpdTrunc}),
and later by constructing its final coalgebra (\cref{ssec:FMGpdLim}, \cref{thm:FMGpdLim}).

\todo[inline]{%
  We should cite \cite{Piceghello2021} here.
  He defines the free symmetric monoidal groupoid on a type
  as a HITs (\enquote{s-lists}).
}

First, recall one way of defining \enquote{finite} sets in HoTT:

\begin{definition}
  A type $Y$ is called \emph{(Bishop-) finite} if
  \[
    \Op{isFinSet} Y \DefEq
      \sum (n : ℕ)\Where \PropTrunc{Y ≃ \Fin n}
  \]
  holds,
  and we denote the collection of such types by
  \[
    \Op{FinSet} \DefEq
      \sum (Y : \Type)\Where \Op{isFinSet} Y
  \]
  The underlying type of a $\FinSet$ is accessed via the first projection $\langle\Blank\rangle : \FinSet → \Type$.
\end{definition}

It is easy to show that $\Op{isFinSet}$ is a proposition and that any type satisfying the predicate is a set.
It follows that $\Op{FinSet}$ forms a groupoid.
Note that $\Op{FinSet}$ is a \emph{large} type, i.e. it lives in the successor universe of the types that it ranges over.
From this, we can define a \enquote{tote}, which we think of as a finite collections of things in $X$:
\begin{definition}
  A large type, $\Tote : \Type_\ell → \Type_{\ell + 1}$:
  \begin{align*}
    \Tote X
      \mathbin{=_{\mathrm{df}}}
      \sum\,\Var{B} : \operatorname{FinSet}\Where B \to X,
  \end{align*}
\end{definition}

In general, $\Tote X$ is at least a groupoid.

\begin{proposition}
  If $X$ is a groupoid, then $\Tote X$ is a groupoid.
\end{proposition}
\begin{proof}
  By assumption $X$ is a groupoid,
  so the function type $\langle Y \rangle → X$ is a groupoid
  for any $Y : \FinSet$.
  But $\FinSet$ is also a groupoid, therefore the entire $\Sigma$-type is a groupoid.
\end{proof}

Note that $\Tote$ is again a large type family.
This makes it unsuitable for iteration, as the assignment $(\lambda n\Where \Tote^n \mathsf{1})$
of $n$-fold applications of the family is not typeable in a universe of bounded size.
We will later\todo{\cref{def:Bag}} introduce an equivalent, small family $\Bag$ that avoids this issue, following \cite{Finster2021}.

\todo[inline]{%
  Relating $\Tote$ back to finite multisets:
}

\begin{theorem}\label{thm:FMSetOfFMGpdTrunc}
  Set-truncating a tote yields a finite multiset.
  For any type $X$,
  \[
    \SetTrunc{\Tote X} ≃ \FMSet X
  \]
\end{theorem}
\begin{proof}
  The proof proceeds by constructing an explicit isomorphism $\FMSet X \cong \SetTrunc{\Tote X}$.
  The forward direction is unproblematic, while the inverse function has to be carefully defined with the
  homotopy levels of the involved types in mind.

  \newcommand*{\ToTote}{\Op{toTote}}
  \newcommand*{\ToFMSet}{\Op{toFMSet}}

  To construct $\ToTote : \FMSet X → \SetTrunc{\Tote X}$, by the induction principle
  $\Elim{\FMSet}$, it is enough to give a function
  $f : \forall \{\Var{sz}\}\Where (\Fin \Var{sz} → X) → \SetTrunc{\Tote X}$
  and to show that it respects $(\sim)$.
  Let $f(v) \DefEq \SetTruncCon[(\Fin \Var{sz} , v)]$, since $\Fin \Var{sz}$ is obviously a finite set.
  To prove that $v \sim w$ implies $f(v) = f(w)$, we note that the conclusion is a proposition,
  thus by induction on $v \sim w$ obtain a permutation $\sigma$ such that $r : v \circ \sigma = w$.
  By univalence, $\Op{ua} \sigma : \Fin \Var{sz} = \Fin \Var{sz}$,
  and transporting $r$ along this path yields
  $
    p : (\Fin\Var{sz} , v) = (\Fin\Var{sz}, w)
  $.
  Then $\Op{cong} \SetTruncCon\, p : f(v) = f(w)$ as desired.

  In the other direction $\ToFMSet : \SetTrunc{\Tote X} → \FMSet X$
  is define by set-truncation elimination, and it is enough to provide
  \begin{align*}
    &g : \Tote{X} → \FMSet X \\
    &g\, (Y, \Var{sz}, e, v) \DefEq (\Var{sz} , g')
  \end{align*}
  where $Y$ is a finite set of size $\Var{sz}$ with $e : \PropTrunc{Y ≃ \Fin\Var{sz}}$ and $v : Y → X$.
  We would like to use a composition $\Fin\Var{sz} → Y → X$ to find
  $g' : \SetQuot[(\Fin\Var{sz} → X)][\mathord{\sim}]$,
  but one cannot do so because $e$ is propositionally truncated.
  Using induction directly on $e$ cannot work, since $g'$ lives in a set.
  We can however show that
  \begin{align*}
    &\operatorname{\mathsf{from-equiv}} : (Y ≃ \Fin\Var{sz}) → \SetQuot[(\Fin\Var{sz} → X)][\mathord{\sim}] \\
    &\operatorname{\mathsf{from-equiv}} \alpha \DefEq \SetQuotCon[v \circ \alpha]
  \end{align*}
  is \emph{weakly-constant} in the sense of \cite{Kraus2017} to obtain $g'$ from $e$ using the elimination
  principle of \cite[{Corollary~2}]{Capriotti2015}.
  Proving $\ToFMSet \circ \ToTote = \Op{id}$ is straightforward.
  Proving $\ToTote \circ \ToFMSet = \Op{id}$ reduces to showing that $v \circ \alpha \sim v$ for any $v : Y → X$ and $\alpha : \Fin n ≃ Y$.
\end{proof}

\begin{definition}\label{def:Bag}
  A small type:
  \begin{align*}
    \Bag X
      \mathbin{=_{\mathrm{df}}}
      \sum\,\Var{x} : \operatorname{\mathsf{Bij}}\Where
        \langle x \rangle \to X,
  \end{align*}
  We abbreviate $\BagCon{v} \DefEq (x, v) : \Bag X$ if $x$ can be inferred from context.
\end{definition}

\paragraph{Comparing $\Op{List}$ and $\Bag$}\todo{Fit this comparison in somewhere}
Note that $\Bag$ is somewhat analogous to $\Op{List}$:
Given a type $A$ and a family $B$ over it, they both match ${\sum (a : A)\Where B(a) → X}$:
we have $A = ℕ$ and $B = \Fin$ for $\Op{List}$,
and $A = \Bij$ and $B = \BagCon{\Blank}$ for $\Bag$.
In particular, $\SetTrunc{\Bij} ≃ ℕ$.

\subsection{As lists up to permutation}

\begin{theorem}
  For any $X$, the types $\FMSet X$, $\M X$ and $\operatorname{PList} X$ are equivalent.
\end{theorem}

\section{The Final Coalgebra}

Describe limits in general.
Maybe compare final coalgebras and corecursive algebras which are fixpoints.

\subsection{Final Coalgebras as an \ensuremath{\omega}-Limit in Set}

\todo[inline]{%
    Surjectivity works (under the assumption of countable(?) choice).
}

\begin{theorem}\label{thm:InjPresImpliesLLPO}
  The function
  $\Op{pres}\colon
      \FMSet (\lim_{n < \omega} \FMSet^n 1)
      \to
      \lim_{n < \omega} (\FMSet^{n+1} 1)$
  is surjective,
  but its injectivity is equivalent to the \emph{lesser limited principle of omniscience}, \LLPO.
\end{theorem}
\LLPO{} \cite[{Ch.\@ 1}]{Bridges1987} is a weak version of the law
of the excluded middle, and it is not provable from intuitionistic
axioms alone.  It states that, given an infinite stream of boolean
values that yields $\True$ in at most one position, one can decide
whether all even or all odd positions are $\False$.

\begin{lemma}\label{lem:DiagLimCaseAnalysis}
  For any $x, a, b : L_{\FMSet}$ and $\Var{ys} : ℕ → L_{\FMSet}$,
  if $\Op{diag} : ∀ n\Where x_n = \Var{ys}(n)_n$,
  then
  \[
    ∀ n\Where (\Var{ys}(n) = a) → (\Var{ys}(1+n) = b) → (!^n (a_{1+n}) = b_n)
  \]
\end{lemma}
\begin{proof}
  \begin{align*}
    \Bang^n(a_{1+n})
      &= a_n \\
      &= \Var{ys}(n)_n \\
      &= x_n \\
      &= \Bang^n(x_{1+n}) \\
      &= \Bang^n(\Var{ys}(1+n)_{1+n}) \\
      &= \Bang^n(b_{1+n}) \\
      &= b_n
  \end{align*}
\end{proof}

\begin{lemma}\label{lem:InjPresImpliesComplete}
  If $\Op{pres}$ is injective, then $L_{\FMSet}$ is complete.
\end{lemma}
\begin{proof}
  Given a limit $\ell : L_{\FMSet}$\todo{We're implicitly working with the HIT-definition here},
  denote by $\ell_n : L_{\FMSet} → \FMSet^n 1$ the projection maps out of the limit.

  Assume $x, y_1, y_2 : L_{\FMSet}$, $\Var{ys} : ℕ → L_{\FMSet}$,
  $\Op{split} : ∀ n\Where (\Var{ys}_n = y_1) \mathrel{+} (\Var{ys}_n = y_2)$
  and $\Op{diag} : ∀ n\Where x_n = \Var{ys}_n(n)$.
  The goal is to show $\PropTruncCon[x = y_1 \mathrel{+} x = y_2]$.
  From $\Op{split}$ we define the complement of $\Var{ys}$,
  \[
    \bar{\Var{ys}}(n)\DefEq
      \begin{cases}
        y_2 & \text{if}~\Var{ys}_n(n) = y_1 \\
        y_1 & \text{if}~\Var{ys}_n(n) = y_2 \\
      \end{cases}
  \]
  We show that the diagonal of $\bar{\Var{ys}}$ too has the limit-property,
  i.e.\@ that
  \[
    ∀ n\Where
      \Bang^n (\bar{\Var{vs}}(1+n)_{1+n}) = \bar{\Var{ys}}(n)_n
      \Step{step:IsLimYsComplement}
  \]
  For this, fix $n$ and check the four cases generated by inspecting $\Op{split} n$
  and $\Op{split}{} (1+n)$.
  In one case, \cref{step:IsLimYsComplement} reduces to the limit-property of $y_1$,
  in another to that of $y_2$ and in the remaining cases to \cref{lem:DiagLimCaseAnalysis}.
  Call the resulting limit $\bar{x} : L_{\FMSet}$.

  Write $\BagCon{x, y} \DefEq \eta\, x \oplus \eta\, y : \FMSet X$.
  Then for all $n$, we know that
  \begin{align}\Step{step:ppc}
    \BagCon{\Var{ys}_n, \bar{\Var{ys}}_n} = \BagCon{y_1, y_2}
  \end{align}
  either by $\Op{refl}$ or $\Op{\oplus comm}$, depending on $\Op{split} n$.
  Then\todo{better wording}
  \[
    \Op{pres} \BagCon{x, \bar{x}}\, n
      = \BagCon{x_n , \bar{x}_n}
      = \BagCon{\Var{ys}(n)_n , \bar{\Var{ys}}(n)_n}
      = \Op{pres} \BagCon{\Var{ys}(n) , \bar{\Var{ys}}(n)}\, n
      = \Op{pres} \BagCon{y_1 , y_2}\, n
  \]
  Therefore, $\Op{pres}(\BagCon{x, \bar{x}}) = \Op{pres}(\BagCon{y_1, y_2})$.
  From injectivity of $\Op{pres}$ it follows that
  \[
    x \in \BagCon{x, \bar{x}} = \BagCon{y_1, y_2}
  \]
  which means that merely $x = y_1$ or $x = y_2$.
\end{proof}

\begin{proof}[\cref{thm:InjPresImpliesLLPO}]
  \Cref{lem:InjPresImpliesComplete} shows that $L_{\FMSet}$ is complete,
  which implies injectivity (\cite[{Theorem~7}]{Veltri2021}).
\end{proof}

\subsection{Final Coalgebras as an $\omega$-Limit in Groupoids}\label{ssec:FMGpdLim}

\begin{theorem}\label{thm:FMGpdLim}
  Let $L_{\Bag} \DefEq \lim_{n < \omega} \Bag^n 1$.
  The limit-preservation map $\operatorname{pres}$ is an equivalence of groupoids.
  In particular, the limit $L_{\Bag}$ is a fixpoint of $\Bag$ and its final coalgebra.
\end{theorem}

\subsection{Truncating the Groupoid Construction}

Compare the set-truncation of the groupoid construction
and the set-level definition.

\begin{theorem}\label{thm:FMSetFixpointOfTrunc}
  The set-truncation of $\Lim{\Bag}$ is a fixpoint of $\FMSet$, i.e.\@
  there is an equivalence $\FMSet \SetTrunc{\Lim{\Bag}} ≃ \SetTrunc{\Lim{\Bag}}$.
\end{theorem}
\begin{proof}
  The equivalence is obtained from the composition
  \begin{align}
    \FMSet \SetTrunc{\Lim{\Bag}}
      &≃ \FMSet \Lim{\Bag}          \Step{thm:FMSetFixpointOfTrunc:step1} \\
      &≃ \SetTrunc{\Bag \Lim{\Bag}} \Step{thm:FMSetFixpointOfTrunc:step2} \\
      &≃ \SetTrunc{\Lim{\Bag}}      \Step{thm:FMSetFixpointOfTrunc:step3}
  \end{align}
  where \eqref{thm:FMSetFixpointOfTrunc:step1} is invariance of $\FMSet$ under set-truncation
  (\cref{thm:FMSetSetTruncInvariant}),
  \cref{thm:FMSetFixpointOfTrunc:step2} is \cref{thm:FMSetOfFMGpdTrunc}
  and \eqref{thm:FMSetFixpointOfTrunc:step3} follows since $\Lim{\Bag}$ is a limit (\cref{thm:FMGpdLim}).
\end{proof}

Note that the above fixpoint is not necessarily the \emph{largest} fixpoint.
\todo[inline,caption={Choice for largest fixpoint}]{%
  \emph{(Unproven)}
  Only assuming the (full) axiom of choice this is the case, i.e.\@ yields a final coalgebra.
  }

\section{Discussion: Alternatives}
\subsection{Outlook: Generalization to Analytic Functors}

\begin{itemize}
    \item Hint at a definition of analytic functors using
        the above definitions (pick a subgroupoid of Bij etc.)
    \item Question:
        Does this definition \emph{weakly preserve pullbacks}?
        This would be the classical definition.
\end{itemize}

\subsection{Using Coinductive Types?}

\begin{itemize}
    \item It is not clear that Agda's coinductive types
        interacts well with HITs.
    \item This might not work with the groupoid-definition:
        Either we use HITs (possibly inconsistent, for the small Bij),
        or the construction ends up being too large.
    \item
        Another approach: Quotient the (entire) final coalgebra of lists.
\end{itemize}

\section{Conclusions}

\subsubsection{Acknowledgements}

\nocite{*}

\bibliographystyle{splncs04}
\bibliography{Multiset}
\end{document}
