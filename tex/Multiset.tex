% This is samplepaper.tex, a sample chapter demonstrating the
% LLNCS macro package for Springer Computer Science proceedings;
% Version 2.21 of 2022/01/12
%
\documentclass[runningheads]{llncs}
%
\usepackage[T1]{fontenc}
% T1 fonts will be used to generate the final print and online PDFs,
% so please use T1 fonts in your manuscript whenever possible.
% Other font encondings may result in incorrect characters.
%
\usepackage{multiset}

\usepackage{graphicx}
\usepackage{amsmath}
\usepackage{amsfonts}
\usepackage{bussproofs}

\usepackage{textcomp}
\usepackage{newunicodechar}

\usepackage{todonotes}

% Used for displaying a sample figure. If possible, figure files should
% be included in EPS format.
%
% If you use the hyperref package, please uncomment the following two lines
% to display URLs in blue roman font according to Springer's eBook style:
%\usepackage{color}
%\renewcommand\UrlFont{\color{blue}\rmfamily}
%

\usepackage{latex/agda}

\usepackage{fancyvrb}
\DefineVerbatimEnvironment
  {code}{Verbatim}
  {}
\begin{document}
\maketitle              % typeset the header of the contribution

\begin{abstract}
The abstract should briefly summarize the contents of the paper in
150--250 words.

\keywords{First keyword  \and Second keyword \and Another keyword.}
\end{abstract}

\section{Introduction}

\section{Type Theory and Cubical Agda}

\section{The Finite Bag Functor}

The action of the bag functor on a type $X$ can be encoded as higher inductive
types in various ways, two of which we present here.
The first is as the type of lists set-quotiented by an \emph{up to permutation}-relation,
the second being the free commutative monoid on $X$.


\subsection{As a HIT}

Given a type $X$, the free commutative monoid $(\M X, \Union, \Empty)$ is
defined as the higher inductive type induced by the following rules:
\begin{itemize}
  \item Point constructors:
    \begin{center}
      \hspace*{\fill}
        \AxiomC{$\vphantom{X}$}
        \UnaryInfC{$\Empty : \M{X}$}
        \DisplayProof
      \hfill
        \AxiomC{$x : X$}
        \UnaryInfC{$\Singl x : \M X$}
        \DisplayProof
      \hfill
        \hspace{10pt}
        \AxiomC{$xs, ys : \M X$}
        \UnaryInfC{$\Var{xs} \Union \Var{ys} : \M X$}
        \DisplayProof
      \hspace*{\fill}
    \end{center}
  \item Monoid axioms and commutativity:
    \begin{center}
      \hspace*{\fill}
        \AxiomC{$\Var{xs} : \M X$}
        \UnaryInfC{$\mathsf{unit} : \Empty \Union \Var{xs} = \Var{xs}$}
        \DisplayProof
      \hfill
        \AxiomC{$\Var{xs}, \Var{ys}, \Var{xs} : \M X$}
        \UnaryInfC{
          $\mathsf{assoc} : \Var{xs}\Union(\Var{ys}\Union\Var{zs}) = (\Var{xs}\Union\Var{ys})\Union\Var{zs}$
        }
        \DisplayProof
      \hspace*{\fill}
      \\[6pt]
      \hspace*{\fill}
        \AxiomC{$\Var{xs}, \Var{ys} : \M X$}
        \UnaryInfC{
          $\mathsf{comm} : \Var{xs}\Union\Var{ys} = \Var{ys}\Union\Var{xs}$
        }
        \DisplayProof
      \hspace*{\fill}
    \end{center}
  \item set truncation:
    \begin{center}
      \hspace*{\fill}
        \AxiomC{$\Var{xs}, \Var{ys} : \M X$}
        \AxiomC{$p, q : \Var{xs} = \Var{ys}$}
        \BinaryInfC{
          $\mathsf{trunc} : p = q$
        }
        \DisplayProof
      \hspace*{\fill}
    \end{center}
\end{itemize}

\subsection{As a Set Quotient}

\begin{definition}
  The type of finite multisets over a type $X$ is the type
  \begin{align*}
    \FMSet X
      \mathbin{=_{\mathrm{df}}}
      \sum\,\Var{sz} : ℕ\Where
        (\Fin \Var{sz} \to X) \mathbin{/_{\!2}} \mathrel{\sim},
  \end{align*}
  where $\sim$ is the proposition-valued relation defined as
  \begin{align*}
    \textunderscore\sim\textunderscore &: ∀ \{n\}\Where (v\, w : \Fin n → X) → \Type \\
    v \sim w &\mathrel{=_{\mathsf{df}}}
      ∃ \sigma : \Fin n ≃ \Fin n \Where
        \operatorname{\mathsf{PathP}} (\lambda i\Where \operatorname{ua}(\sigma)\, i → X)\, v\, w
  \end{align*}
\end{definition}

\subsubsection{Finite Choice for Sets}

\begin{lemma}
  Set truncation distributes over finite families of types.
  For any $n : ℕ$ and type family $Y : \Fin n \to \Type$,
  there is an equivalence
  \[
    \Op{box} :
    ((k : \Fin n) → \SetTrunc{\mathop{Y\/} k})
    ≃
    {\SetTrunc{(k : \Fin n) → \mathop{Y\/} k}}
  \]
\end{lemma}

\begin{definition}
  The principle of finite choice can be defined in terms of $\Op{box}$:
  \begin{align*}
    \Op{elim}_{\Op{fin}} &: \forall n\Where \\
      &\to \{B : (\Fin n \to \SetTrunc{X}) \to \Type\} \\
      &\to (\forall v\Where \operatorname{isSet} (B\, v)) \\
      &\to ((\mathsf{choice} : \Fin n \to X) \to B\, (\SetTruncCon{} \circ \mathsf{choice})) \\
      &\to (v : \Fin n \to \SetTrunc{X}) \to B\, v)
  \end{align*}
  It is defined by applying $\mathsf{choice}$ to the term obtained from
  set-truncation elimination on $\operatorname{box} v$.
  It comes with a computational rule
  \begin{align*}
    \Op{elim}^\beta_{\Op{fin}} : (v : \Fin n → X)
      → \Op{elim}_{\Op{fin}} (\SetTruncCon{} \circ v) = \operatorname{\mathsf{choice}} v
  \end{align*}
\end{definition}

\begin{theorem}
  Finite multisets are invariant under set-truncation:
  \begin{equation}
    \FMSet \SetTrunc{X} ≃ \FMSet X
  \end{equation}
\end{theorem}
\begin{proof}
  \begin{enumerate}
    \item Inverse is given by $\operatorname{map} \SetTruncCon{} : \FMSet X → \FMSet \SetTrunc{X}$.
    \item Define the inverse by using
      \[
        \Op{requot} : ∀ \{n\}\Where
          (\Fin n → \SetTrunc{X})
          → (\Fin n → X) \SetQuot{} {\sim_n},
      \]
      defined via $\Op{elim}_{\Op{fin}}$.
  \end{enumerate}
\end{proof}

\subsection{As a Groupoid}
Definition as a $\Sigma$-type,
but indexed by a groupoid instead
of FinSet.

\begin{itemize}
    \item Compare large/small definition.
\end{itemize}
\begin{definition}
  \begin{align*}
    \FMGpd X
      \mathbin{=_{\mathrm{df}}}
      \sum\,\Var{x} : \operatorname{\mathsf{Bij}}\Where
        \langle x \rangle \to X,
  \end{align*}
\end{definition}


In general, $\FMSet X$ is at least a groupoid.

\begin{theorem}
  Set-truncating a bag yields a finite multiset.
  For any type $X$,
  \[
    \SetTrunc{\FMGpd X} ≃ \FMSet X
  \]
\end{theorem}

\section{The Final Coalgebra}

Describe limits in general.
Maybe compare final coalgebras and corecursive algebras which are fixpoints.

\subsection{Final Coalgebras as an $\omega$-Limit in Set}

\todo[inline]{%
    Surjectivity works (under the assumption of countable(?) choice).
}

\begin{theorem}
    Injectivity of the limit-preservation map
    \[
        \operatorname{pres}_{\Lim (\Chain \M)}\colon
            \M (\Lim (\Chain \M))
            \to
            \Lim (\Shift(\Chain\M))
    \]
    implies \LLPO.
\end{theorem}

\todo[inline]{%
    Think about the converse statement,
    does \LLPO already imply injectivity?
}

\subsection{Final Coalgebras as an $\omega$-Limit in Groupoids}

\subsection{Truncating the Groupoid Construction}

Compare the set-truncation of the groupoid construction
and the set-level definition.

\begin{itemize}
    \item
        Show that the set-truncation of the limit in
        groupoids stays a fixpoint for the set-level functor.
    \item
        \emph{(Unproven)}
        Only assuming the (full) axiom of choice this is the
        largest fixpoint, i.e.\@ yields a final coalgebra.
\end{itemize}

\section{Discussion: Alternatives}
\subsection{Outlook: Generalization to Analytic Functors}

\begin{itemize}
    \item Hint at a definition of analytic functors using
        the above definitions (pick a subgroupoid of Bij etc.)
    \item Question:
        Does this definition \emph{weakly preserve pullbacks}?
        This would be the classical definition.
\end{itemize}

\subsection{Using Coinductive Types?}

\begin{itemize}
    \item It is not clear that Agda's coinductive types
        interacts well with HITs.
    \item This might not work with the groupoid-definition:
        Either we use HITs (possibly inconsistent, for the small Bij),
        or the construction ends up being too large.
    \item
        Another approach: Quotient the (entire) final coalgebra of lists.
\end{itemize}

\section{Conclusions}

%
% the environments 'definition', 'lemma', 'proposition', 'corollary',
% 'remark', and 'example' are defined in the LLNCS documentclass as well.
%
\subsubsection{Acknowledgements}

\nocite{*}

\bibliographystyle{splncs04}
\bibliography{Multiset}
\end{document}
